\section{Plan of Work}

We recognize that \name\ will be a significant expense for the
balloon program.  In order to ensure that we are on-track with our
technology and on-schedule to achieve our goals, we will have
milestones and reviews by outside experts.  We will organize external
reviews of the two most critical and difficult aspects of the
instrument: the spectrometer optics (including configuration and
layout in the cryostat), and the detectors.

\subsection{Schedule}

\noindent{\bf GY1: Oct 2013 - Sep 2014:} Our first year will be
dominated by design and testing in preparation for Preliminary Design
Review (PDR) in May and a Critical Design Review (CDR) in September.
There will be separate PDRs and CDRs for the optics and detectors.
While this is an agressive schedule for these reviews, we need to
prepare for a launch in the 4th year of the work.

JPL has already done a considerable amount of work on the detectors.
%Their schedule of work is outlined in Figure \ref{fig:JPLSchedule}
An initial optics design has been completed. However, it will need to
be optimized and the details of laying out the optics, detectors, and
fridges in the cryostat which will need to be finalized.

% \begin{figure}[h]
%     \begin{minipage}{6.5in}
%       \begin{center}
% 	\includegraphics[width=6.5in]{ICARISSchedule2012.pdf}
%       \end{center}
%     \end{minipage}
%     \caption {\small JPL Schedule of Work}
% \label{fig:JPLSchedule}
% \end{figure}

\noindent{\bf GY2 - 3: Oct 2014- Sep 2016:} These years will be
dominated by building and testing all of the components of \name.
JPL will begin the production of the first detectors as well as the
detector package.  Penn will work with the contractors to complete the
telescope and gondola.  In addition we will construct the receiver,
cryogenics, and cold optics.  In the summer of 2016 we will integrate
the instrument and prepare for mission readiness review.

\noindent{\bf GY4: Oct 2016 - Sep 2017:} The first flight of \name\
is targeted for June 2017.  Data reduction software will be in place,
and reduction will begin promptly upon data acquisition.

\noindent{\bf GY5: Oct 2017 - Sep 2018:} The final year of the grant
is devoted to analysis of the data and a planned second flight in June 2017.  We have a strong track record in
analysis and releasing our data promptly to the community through the
NASA IPAC archive.\footnote{See, for example:\\ {\tt
http://irsa.ipac.caltech.edu/Missions/bolocam.html} and\\ {\tt
http://irsa.ipac.caltech.edu/Missions/blast.html}}

We have considered the reduction of risk throughout the process,
largely by drawing on the proven \blast\ heritage for the gondola and
pointing.  We also consider here
several contingencies and options for descope.  The \name
spectrometer design is highly modular.  This allows for the possiblity
of excluding one or more modules from the flight instrument if there
is a problem with the detectors or optics design.  The impact on the
science would be a reduction of the redshift range accessible.  

% The
% Antarctic flight is planned for GY4 Q4 (June 2017).  This leaves
% 1.5 years after recovery for data analysis.  However, should the
% schedule slip, a flight occurring in January 2017 would still provide
% 8 months for data analysis within the current grant.  As the
% additional year would also allow further development of the software,
% this should be sufficient, although we still target an earlier flight.

\subsection{Division of Labor and Personnel} 

To achieve the ambitious goals of this program, we have assembled a
group of scientists with the wide range of backgrounds necessary to
design, build and fly the instrument within the time scale described.
%,
%including key contributions from our international collaborators.  
Our
team includes world leaders in advanced submillimeter detectors,
background-limited (sub)millimeter spectroscopy, and scientific
ballooning.  Our team includes veterans from multiple balloon payloads
(notably \blast).% as well as suborbital rocket missions.  
We also have
extensive experience with millimeter-wave spectrometers (\zspec),
submillimeter receivers (ACT), and detector development (BLISS).  
%We
%have included theorists among our instrumental team to guide the
%design and speed the analysis of data.

The project planning is a group effort coordinated by the PI at the
University of Pennsylvania through frequent telecons and WWW-based
information exchange.  The University of Pennsylvania (Aguirre,
Devlin) will build the receiver, telescope and gondola.  JPL
(Bradford) will supply the detectors and detector mounts.  
%NIST will
%supply the cold multiplexing electronics.  The University of British
%Columbia (Halpern) will provide the multiplexing hardware for the
%detectors as well as crucial support for flight planning.  Cardiff
%(Pascale) will provide flight campaign and data analysis support, as
%well as cryogenic filters.  Cornell (Stacey) will provide oversight on
%scientific requirements.  Irvine (Cooray) will lead the theoretical
%work.  
All collaborators will share in the data analysis.  Close
collaboration between the experimentalists, theorists and observers in
our team will continue throughout the flight planning, field and
source selection, and the analysis of the data.  The entire team has
been involved with the successful analysis of several large, complex
CMB and submillimeter datasets and has an excellent history of prompt
publication and distribution of data.

%\subsubsection{Penn}
%
%Aguirre was PI (and C.~M.~Bradford co-PI) of NSF AST 0807990,
%``Broadband Millimeter Spectroscopy with \zspec: Molecular Diagnostics
%of Local ULIRGs and Redshifts for Submillimeter Galaxies'', which
%expired in June 2011.  This grant produced detailed studies of the
%local (U)LIRGs M82 \Citep{2010ApJ...722..668N} and NGC1068
%\citep{2011ApJ...731...83K}, as well as redshifts for high-$z$
%galaxies from the {\em Herschel} ATLAS and HerMES surveys
%\Citep{2010arXiv1009.5983L,2011ApJ...733...29S,2011ApJ...732L..35C,2011ApJ...738..125G}.
%These redshifts were essential in establishing the strongly lensed
%nature of the brightest {\em Herschel} sources, as reported in
%\citet{2010Sci...330..800N}.  Other exciting discoveries from Z-Spec
%also included the largest known mass of water around the quasar
%APM08279+5255 \citep{2011ApJ...741L..37B} and the discovery of water
%in other high-redshift star-forming galaxies
%\citep{2011A&A...530L...3O}.  
%
%Aguirre is also currently co-PI of NSF
%AST 1125558 ``Collaborative Research: Precision Array for Probing the
%Epoch of Reionization (PAPER)''.  PAPER is a dedicated attempt to
%detect the highly redshifted 21 cm emission from the epoch of
%reionization.
%
%The array
%is being constructed in the Karoo Desert of South Africa, near the
%South African SKA site, and since beginning operations there in
%October 2009, has reached 64 operational antennas, half its planned
%size.  Data of excellent quality is pouring in, and PAPER is on track
%for full science data in late 2012 or early 2013.  Aguirre has one
%student supported under this grant working on PAPER data
%analysis. Recent results from PAPER include a new southern sky point
%source catalog \citep{2011ApJ...734L..34J}, compiled by the PI's first
%graduate student, who defended in July 2011.  Considerable work has
%also gone into novel approaches to array configuration and power
%spectrum analysis \citep{2011arXiv1103.2135P}.  
%

%Co-I Devlin was the PI of \blast, together with co-I's Pascale and
%Halpern.  In December of 2006 \blast\ had an 11-day LDB flight in
%Antarctica.  \blast\ demonstrated our successful integration of the
%sensitive detector arrays and readout electronics, precision optics,
%accurate pointing system, careful flight planning, and finally,
%intensive data analysis.  BLAST has a produced a tremendous body of
%work, with more than twenty published papers (including one in Nature
%\Citep{devlin09}).  BLAST also produced a number of results on
%submillimeter galaxies which anticipated \herschel, including
%measurements of the galaxies' clustering \Citep{viero09}, number
%counts \Citep{patanchon09}, star formation history \Citep{pascale09},
%redshift distribution and luminosity function \Citep{eales09}, as well
%as the far-IR radio correlation \Citep{ivison10}, and the detection of
%strongly lensed sources by clusters \Citep{rex09}.  \blast\ was a
%fantastically successful project.  We hope to extend its legacy with
%\name.
%
%Toronto (Netterfield) will provide and support most of the updated
%electronics and data acquisition system.  They will also continue to
%be the main supporters of the flight software.  

 

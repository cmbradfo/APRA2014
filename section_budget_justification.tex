\begin{center}
{\bf \large Budget Justification}
\end{center}

This proposal asks for funds to design, build, and fly \name\ and
analyze its data.  
%Like \blast, we will accomplish much of what can be
%done from a satellite at a small fraction of the cost.  
We already have extensive infrastructure from \blast\ for support in the field.  The \name\ gondola and flight electronics are clones of the proven \blast\ design.  We believe it will take three years to design and build the payload
with overnight North American flights in the fourth and fifth year, with analysis promptly after the flights.  The total amount of data will be modest, and flying later will allow the reduction pipeline to be more fully developed.

A large fraction of our budget goes to JPL to design, develop and
build the detectors.  The \name\ detector development program will make possible a future balloon mission competitive with satellites, as well as  
provide dividends for future space missions and benefit the
community generally.  The Penn budget includes JPL as a subcontract;
however as a NASA center, these funds are allocated directly.  The JPL
budget is included as a separate spreadsheet.

% We have also included a subcontract to A. Cooray at University of
% California Irvine to provide support for the theoretical calculations
% necessary for successful scientific returns, and for assistance with
% the data analysis to ensure sufficient manpower to complete the
% analysis within the lifetime of this grant.

% \name\ will rely on the commitment and support from our foriegn
% collaborators.  %In particular, Enzo Pascale (Cardiff), and Mark
% Halpern (UBC) will be leading efforts at their home institutions.
% They are some of the most experienced people in scientific ballooning.
% Their expertise is highly sought after and we are fortunate to have
% them on our experiment.  In addition, they bring substantial resources
% in terms of people and equipment.  However, our plan does not rely on
% them receiving direct funding for
% \name.

\begin{itemize}
\item 
{\bf Personnel} We have a number of very experienced people working on
\name, and will train new postdocs and graduate students.

\begin{itemize} 

\item
Jeff Klein has functioned as the project scientist for \blast\ since
its inception.  He is universally accepted within the collaboration as
an indispensable member of the team.  His knowledge base supports the
entire project.  Through his work on ACT he has also become an expert
with the TES detectors and most importanly the multiplexing
electronics. He is supported 2 months/year on \name\ during the
final three years of the grant to support integration and field work.

\item
Simon Dicker has been a scientist at Penn for more than 10 years.  He
designed, built, and fielded the first array of bolometers on the
Green Bank Telescope. He has extensive experience with optical design
and TES detectors.  He is supported during the initial portion of the
award to work on optical and cryogenic design, and has already begun
designing the spectrometer and telescope optics.

\item
We will hire two postdocs at Penn to help design and build the many
new components for \name, as well as a postdoc through the JPL
subcontract
%, and on e through the Irvine contract.  
We have included
the biographical sketch of Steve Hailey-Dunsheath, who has 
%postdocs who have 
expressed an interest
in the proposal and contributed to it.
%(Lupu, Murphy, Hailey-Dunsheath,
%Gong, de Bernardis).  These are potential candidates for hire at one
%of Penn, JPL, or Irvine.

\item
There will be two new graduate students at Penn.  With the help of our
experienced staff and postdocs, these students will be coming up to
speed on the \name\ instrument to help support the next generation
of experiments.  They will also work on specialized data planning and
analysis.  Aguirre currently has one student who will help train the new students: Bade Uzgil, partially supported by a NASA GSRP has been
doing theoretical calculations in intensity mapping.

\item
Undergraduates have been involved with all of our projects (over 50 in
the last 10 years).  They have worked closely with our team to help to
build and support the instruments.  Most move on to graduate school.  Some
are even working on balloon payloads for other projects.  Support is
requested for one 40 hour-per-week undergraduate each summer.

\item
There is partial summer salary support for Aguirre and Devlin.
The JPL subcontract provides partial support to Bradford.

\item
For calculation of overhead, Modified direct cost = total direct cost -
equipment - tuition

\end{itemize}

\item
{\bf Travel.}  The travel budget for \name\ is dominated by the field travel for two one-month campaigns in Palestine or Ft. Sumner.
%Mission Readiness in Palestine
%(one month), and LDB flights from Antarctica and/or Sweden.  
We have
been in the field several times with our instruments and have experience
conserving resources (such as renting houses instead of
individual rooms in hotels).  
%In addition to the field travel for
%launch and recovery, funds are requested in years 1 - 3 for travel
In addition, there is the necessary travel back-and-forth between JPL and Penn for training and assistance with testing and
integration (cross-pollination of the postdocs and graduate students) and also during years 4 and 5 for collaborative work on data analysis.  We have estimated \$60,000
for domestic travel over the five years.  In addition, \$5,000 in
foreign travel is requested for travel to conferences to present results.

\item
{\bf Lab Supplies.}  This covers miscellaneous hardware and tools
required for construction and testing, as well as flight operations.
A total of \$60,000 over 5 years is requested, with higher amounts in years 3 and 4 for contingency in preparing for the first flight.

\item
{\bf Cryogens.}  These are for flight operations as well as
running multiple laboratory tests.  As much as possible, we will use
an existing closed-cycle cryogenic system (see Facilities and
Resources) to conduct cold tests and thus conserve cryogens.
Estimated costs are based on an inflation-adjusted average of \$10 per liquid liter, and a consumption rate of the cryostat of 100 liters on cooldown and 10 liters per day there after.  Thus, a two-week cooldown costs approximately \$1500, allowing for boiloff and transfer losses.   We request \$12,000 per year (8 two-week cooldowns) for each year we have the cryostat (years 2 - 5)


\item
{\bf Publication.}  
We expect to have a significant number of papers from our flight
within a short period after data collection, and so we have included
the cost of publications years 4 and 5 of the grant.  A total of
\$10,000 is requested, amounting to about four large publications.

\item
{\bf Subcontracts} There is a subcontract to JPL, with budget attached separately. The JPL award will be made directly from NASA.  

\item {\bf Equipment}

\begin{itemize}

\item
{\bf Cryostat.}  We have based the cost on the Z-Spec dewar, which has a 48-hour hold time and ample cryogenic volume for the spectrometer optics.  
It includes the cost for the He3/He4 fridges and magnetic shielding.  The housekeeping thermometry and optics are specified separately.
% The
% internal volume required for \name\ is larger than that for \blast,
% and we have scaled up the price accordingly.  
Total cost is
estimated at \$50,000.
% divided over three years to allow for design,
% major manufacturing, and inevitable modifications.

\item
{\bf Cryostat Electronics.}  This includes cryogenic housekeeping thermometers and cables,  and their readout electronics, based on the \blast\ design and
experience.  No significant change is anticipated over the \blast\
budget, so we have simply reproduced the request, \$60,000.

\item
{\bf Adiabatic Demagnetization Refrigerator (ADR)}.  This provides the final stage of cooling of the detectors to 160 mK.  The cost of \$60,000 is based on a commercially available system quoted by High Precision Devices, Inc (HPD) in Boulder, Colorado. 

% \item
% {\bf Cryogenic Cable Harness.}  This is the cryogenic cabling not
% supplied as part of the JPL subcontract.  These cables will be almost
% exact duplicates of the ones used on ACT.  This will save on design
% costs, but they are still expensive to purchase from TekData in the
% UK, at an estimated cost of \$25,000.

\item
{\bf Pointing Electronics.} This includes all of the electronics to
run the gondola.  There are two computers, power distribution,
pointing sensors and star cameras, and interface boards to the motors and pointing
sensors.  Again, this is based on the \blast\ design, which will be
replicated, at a cost of \$60,000.

\item
{\bf Gondola.}  We will re-use the \blast\ 2010 gondola.  The cost here (\$10,000) is for a new inner frame to support the larger, heavier off-axis mirror.

\item
{\bf Gondola Motors.}  The \blast\ drive motors will need to be
redesigned for the larger mass and moment of the \name\ primary
mirror and cryostat, at an estimated cost of \$30,000.

\item
{\bf Primary Mirror.}  The primary mirror cost is based on a detailed estimate by
Magna Machining using a SolidWorks model of our current design, and including tolerance requirements.  The \$400,000 cost includes material, lightweighting machining, and surface cutting and polishing of {\it two} 2.5-meter off-axis mirrors.  Because the surface cutting is done on a vertical lathe, both mirrors can be cut simultaneously.  The second mirror provides a spare without significantly increasing the cost.

\item
{\bf Secondary Mirror.}  This is for the secondary mirror and
positioning system.  This is a nearly exact clone of the \blast\
design, and is costed accordingly.  Costs include the machining of the
mirror surface and lightweighting for the mirror, as well as the
focus-adjust mount for the mirror, its linear actuators, and the
electronics, estimated at \$50,000.

\item
{\bf Cold Optics.}  This is for the machining of the cold image slicing and relay optics, the diffraction gratings, the blocking filters, their support stuctures, and the mechanism for moving the grating.  The optical design calls for two spectrometer modules, each fed by an image slicer.  
Each spectrometer module has two mirrors (4 total), all of which are powered.  The surface accuracy of these mirrors can be
achieved with a machining process, and the largest of them is $\sim$20
cm.  In addition, there are two blazed diffraction gratings.  Based
on experience with the ZEUS spectrometer, the gratings are available
commercially at an expected cost per grating of approximately
\$15,000, and for mirrors of similar size and accuracy, the typical
cost is \$3,000 - \$5,000, depending on size, averaging about \$4,000.  Each module also requires positioning and alignment structures.  We estimate a combined cost of \$62,000 for both spectrometer modules.  The image slicing optics are much smaller, but require high alignment precision.  Combined with the necessary mounting hardware, we estimate \$25,000 for each slicer, for a total of 
\$112,000.

\item {\bf Lenslet Arrays}  These are the silicon lenses for focusing light on the KIDs.  Four are required, one for each array, from  VeldLaser at \$2000 each (as scaled from our prototype), for a total of \$8,000.

\item 
{\bf Detector Readout Electronics}
This is the frequency multiplexed readout for the kinetic inductance detectors (KIDs).  The cost includes \$14,000 for 4 SiGe amplifiers from Sandy Weinreb at Caltech (\$3.5k each), \$10,000 for 4 sets of cryogenic coaxial cable (\$2.2k for 2 NbTi runs (for $T<4$~K) and \$300 for 2 stainless runs), \$22,000 for the ROACH  electronics (two systems, each handling 2 readout chains).  Each ROACH system consists of: ROACH-1 board (\$2500), XC5VSX95T FPGA (\$2250), iStar chassis (\$515), ADC / DAC combo cards from Rick Raffanti (\$2650), and a readout computer (\$3000) which also handles data storage.  The total cost of the readout system is \$48,000.

% \item
% {\bf Detector Readout Electronics.}  These are the non-cryogenic
% electronics for driving the detector readout SQUID multiplexing
% (``mux'') system.  They will be supplied and supported by Mark
% Halpern's group at University of British Columbia.  The designs are
% based on those in operation on several experiments.  The mux scheme
% uses chip-select within the cryostat.  Each box is capable of reading
% out $32 \times 50 = 1600$ detectors, or enough for one of the $24
% \times 64 = 1536$ spectrometer modules.  The cost for one MCE box is
% \$75,000.  This is a complete system, including power supplies,
% control computer, fiber optics and cables, extension cards,
% MDM-connector breakout cards, and software and firmware support.  In
% addition, there is a one-time, non-recoverable engineering cost of
% developing the new firmware for chip-select addressing of \$8,000.
% The total is thus \$308,000. The budget is allocated so that 25\% of
% the system and its NRE are provided in the first year for the initial
% round of detector construction at JPL, with 25\% in the second year,
% and the remaining 50\% in year 3 prior to full delivery.
% 
% \item
% {\bf NIST Multiplexing Electronics} These are the cryogenic
% first-stage and series array SQUID amplifiers.  \name's 6144 pixels
% will be divided into 94 columns, each composed of 6 11-channel mux
% chips, for a total of $6 \times 94 = 564$ chips. At \$825 per chip,
% this comes to \$465,300.  The 94 columns require 94 series-array
% SQUIDs, Nyquist and bias chips, and Nb magnetic shielding modules.
% The series array SQUIDs are \$28,200, Nyquist and bias chip
% \$53,300, and the modules \$10,600, for a total of \$557,400.  The budget 
% is allocated so that 25\% of the system is provided in the first year
% for the initial round of detector construction at JPL, with 25\% in
% the second year, and the remaining 50\% in year 3 prior to full
% delivery.


% \item
% {\bf Solar Panels.}  We will purchase the same solar panels that we
% used on \blast, at a cost of \$25,000.

\end{itemize}

\end{itemize}

\begin{center}
{\bf \large Facilities, Equipment and Other Resources}
\end{center}
\medskip
\noindent{\bf \large Penn}

%\begin{itemize}

{\it Existing Designs, Instrumentation and Equipment.}  \name\ will
rely heavily on existing designs and instrumentation from \blast, ACT,
and other projects.  Whenever possible we will share equipment with
\blast.  This includes systems such as the star cameras and other pointing 
sensors.  We also have complete laboratory and field supplies that
will be used for \name.\\

{\it Laboratory.} The University of Pennsylvania is equipped with a
high bay facility in which \blast-pol is currently undergoing
integration.  This facility will be sufficient for the integration of
the \name\ gondola.  Between Profs. Aguirre and Devlin there is
approximately 1,800~ft$^2$ of space allocated for this work and can
use 600~ft$^2$ more if necessary.  All of the rooms are plumbed for a
vacuum system and compressor.  The ceilings have hard attachment
points to mount 2-ton hoists.  One of the rooms is accessible by
fork-lift from the loading dock to move large pieces of equipment.\\

{\it Shops.} The physics department machine shop at Penn will be used
to manufacture most of the parts.  A CNC milling machine and lathe are
available with a skilled machinist.  The electronics lab at Penn is
extremely knowledgeable and their services can be made available as
needed.\\

{\it Computation and Data Storage.}  Aguirre maintains the central
computer cluster for PAPER data analysis and archival at Penn.  This
consists of 16 high-speed compute nodes as well as $>100$~TB of
storage.  This facility will be upgraded over the next two years as
part of NSF grant AST 1125558 ``Collaborative Research: Precision
Array for Probing the Epoch of Reionization (PAPER)'', at no cost to
this grant.  The computational needs of \name\ are minor compared to
the resources available from this shared resource.

The following equipment at Penn will be available for use on this
project:

\begin{itemize}

\item 
{\it Dedicated test cryostat} A cryostat with pulse-tube cooler,
$^4$He and $^3$He sorption fridges, and instrumentation will be
available for testing detectors and other cryogenic tests to avoid the
unnecessary use of liquid cryogens in the flight cryostat.

\item
{\it Cryogenic equipment.} Considerable infrastructure for supporting
cryogenic measurements exists and are ready to be dedicated to this
project.

\item 
{\it Infrared Fourier Transform Spectrometer.} This was built at Penn
by undergraduates and graduate students. It operates from 80~GHz to
1.2~THz and will be used to characterize the \name\ bandpass
response.

%\item {\it Other Resources.}  The Cardiff group will provide all of the
%filters needed for this work.  

\item {\it Standard laboratory test equipment} 

\end{itemize}

\pagebreak

\noindent{\bf \large Caltech}

We have access to a 600-square-foot laboratory space which is earmarked for this work.   We have a dedicated cryostat fitted with coaxial leads and cooled to 300 mK with a combination of a 4-K Cryomech pulse tube refrigerator and a Chase $^3$He cooler.   For initial measurements as our new readout is under development,  we also have access to HEMT amplifiers (LNAs), a vector network analyzer (VNA), a 0--40~GHz frequency synthesizer, preamplifiers, IQ demodulators, digitizers and a data acquisition computer.  These components allow the readout of a
single channel using a standard homodyne detection technique. 

For spectral profile measurement, we have both a long-throw ($\delta\nu$=300~MHz) Fourier-transform spectrometer as well as tunable local-oscillator sources for all frequencies throughout our range.

Finally, we have a millimeter / submillimeter wave beam mapper formed from a chopped thermal source on a 2-D raster stage.


%
%
%Dr. Peter Day (PI) maintains two highly instrumented dilution refrigerators dedicated to
%superconducting detector research. The first is a Janis Inc. conventional "wet" dilution
%refrigerator which is instrumented with SQUID amps, microwave LNAs, and an in-situ
%temperature controlled black body which can characterize the performance of both TES and
%MKID type detectors.  A second cryogen free or "dry" dilution refrigerator from Leiden
%Cryogenics has recently been commissioned. This refrigerator is currently operating with six
%sets microwave transmission lines and LNAs for testing of detectors with microwave readouts. In
%addition, the latter unit has a probe insert which can be introduced into cold space of the
%refrigerator from room temperature while operating the refrigerator. This will enable fast
%turnaround measurements.  A network analyzer, microwave frequencies synthesizers and related components are available for device characterization.
%

\paragraph{Superconducting Device Fabrication at the Jet Propulsion
Laboratory Microdevices Laboratory (MDL).}

Dr H.G.\ `Rick' LeDuc and his group at JPL maintains several sputter systems for
the sputtering of high-quality superconducting films including Nb,
Mo, Au, Ti, and Al and SiO2 films.  There are Unaxis Shutteline
chlorine and fluorine inductively-coupled plasma (ICP) etchers in
MDL for etching metal and dielectric films.

%MDL has a 0.25\micron resolution Cannon FPA 300 EX stepper with a 50
%nm alignment tolerance essential for defining the small features in
%our proposed TES bolometer design.  A JEOL 9300 electron beam writer
%is available to fabricate the wire mesh absorber of the bolometer.

Key lithographic tools within the Microdevices Laboratory (MDL)
include a Canon FPA300-EX3 projection stepper and a JEOL JBX-9300 FS
electron beam writer. The stepper has a 0.6 numerical aperture and a
KrF eximer laser light source (248nm) giving a resolution of 0.25
\micron and a high throughput rate. The electron beam writer has a
field emission source and a 100kV accelerating potential producing a
nominal 4nm electron beam diameter and nominal 20nm lines.  These
tools can be operated in a complementary mode in which the highest
resolution patterns are written by the e-beam tool and matched with
lower resolution patterns exposed by the stepper to maximize wafer
throughput.

\paragraph{Computing.} We have access to Caltech Astronomy Data Processing Facility, which hosts and maintains licenses for
various electromagnetic simulation and data analysis software such as L-Edit, HFSS, Sonnet,
ADS, IDL, and Matlab. We have included the cost for accounts and software licensing in our
budget.

%\subsection{MKID Test Facilities at JPL}

% The superconducting bolometer group centered at JPL's Microdevices
% Laboratory (MDL) specializes in the fabrication of low-T$_c$
% superconducting devices.  The group has extensive experience with
% low-T$_c$ materials and tunnel junctions, and is internationally
% recognized as the preeminent group for fabrication of superconducting
% mixer devices, having produced a number of important breakthroughs in
% materials and device technology for these applications.
% 
% The group has state-of-the-art facilities enabling in-house
% fabrication of membrane-isolated TESs from bare Si wafers to working
% devices.  Specific facilities include:\begin{itemize}
% 
% \item{A Tystar low-pressure chemical vapor deposition (LPCVD) system for 
% growing sub-micron $\rm Si_xN_y$ films on Si wafers.}
% 
% \item{Several systems for sputtering high-quality superconducting films of 
% materials such as Nb, Mo, Au, Ti, Al, as well as SiO2 films.}
% 
% \item{Unaxis Shutteline chlorine and fluorine inductively-coupled plasma 
% (ICP) etchers for etching metal and dielectric films.}
% 
% \item{State of the art lithographic tools, including a Canon FPA300-
% EX3 projection stepper and a JEOL JBX-9300 FS electron beam writer.
% The stepper has a 0.6 numerical aperture and a KrF eximer laser light
% source (248~nm) giving a resolution of 0.25~$ \rm\mu m$ and a high
% throughput rate. The electron beam writer has a field emission source
% and a 100~kV accelerating potential producing a nominal 4~nm electron
% beam diameter and nominal 20~nm lines. These tools can be operated in
% a complementary mode in which the highest resolution patterns are
% written by the e-beam tool and matched with lower resolution patterns
% exposed by the stepper to maximize wafer throughput}
% 
% \item{A Surface Technologies Systems deep-trench Si etcher to perform
% back-side anisotropic etching of the bolometers. Front side releases
% can be performed using a XeF$_2$ etcher.}
% 
% \end{itemize}
% 
% The JPL group also makes use of a number of cryogenic test facilities,
% which are available to this proposal at no cost:
% \begin{itemize}
% \item
% {A Janis JDR-100 $^3$He-$^4$He dilution refrigerator (DR) which
% reaches a base temperature of $<10~\rm\,mK$, and includes a
% temperature-controlled blackbody source for optical characterization.
% This cryostat is outfitted with four SQUID controller / readout
% channels, all filtered to eliminate RF injection to enable low-NEP
% testing.}
% 
% \item
% {A Janis Adiabatic Demagnetization Refrigerator (ADR) equipped with
% $8\times 32$-channel time domain NIST SQUID multiplexer (MUX chips at
% 1~K, cryogenic cabling, room-temp control electronics).}
% 
% \item
% {A Cryogen-Free testbed 50 mK testbed with an optical port has been
% procured and is under construction.  It will be compatible with the
% SQUID TDM, and will be used to demonstrate performance of the arrays
% before delivery to Philadelphia.} 
% 
% \end{itemize}

\end{document}